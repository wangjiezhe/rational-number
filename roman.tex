\section{罗马数字}

\begin{frame}\frametitle{罗马数字}
	罗马数字共有7个:\\
	\vspace{0.5em}
	\renewcommand{\arraystretch}{1.5}
	\begin{tabular}{| *{8}{c|}}
		\hline
		罗马数字 & \alert{I} & \alert{V} & \alert{X} & \alert{L} & \alert{C} & \alert{D} & \alert{M} \\ \hline
		数值 & 1 & 5 & 10 & 50 & 100 & 500 & 1000 \\ \hline
	\end{tabular}\pause
	\begin{block}{拼写规则}
		\begin{itemize}
			\item 右加左减
			\item 加线乘千
			\item 数码限制 
		\end{itemize}
	\end{block}
\end{frame}

\begin{frame}\frametitle{罗马数字}
	\begin{block}{右加左减}\pause
		\begin{itemize}[<+- | alert@+>]
			\item 在较大的罗马数字的右边记上较小的罗马数字,表示大数加小数
			\item 在较大的罗马数字的左边记上较小的罗马数字,表示大数减小数
			\item 左减的数字有限制,仅限于I、X、C\\
				比如45不可以写成VL,只能是XLV
			\item 左减时不可跨越一个位值\\
				比如,99不可以用IC(\texttt{100-1})表示,而是用XCIX( \texttt{[100-10]+[10-1]})表示。
		\end{itemize}
	\end{block}
\end{frame}

\begin{frame}\frametitle{罗马数字}
	\begin{block}{加线乘千}\pause
		\begin{itemize}[<+-| alert@+>]
			\item 在罗马数字的上方加上一条横线,表示将这个数乘以1000,即是原数的1000倍
			\item 同理,如果上方有两条横线,即是原数的1000000( $1000^2$)倍
		\end{itemize}
	\end{block}
\end{frame}

\begin{frame}\frametitle{罗马数字}
	\begin{block}{数码限制}\pause
		\begin{itemize}[<+-| alert@+>]
			\item 同一数码最多只能连续出现三次,如40不可表示为XXXX,而要表示为XL
			\item 例外:由于IV是古罗马神话主神朱庇特(即IVPITER,古罗马字母里没有J和U)的首字,因此有时用IIII代替IV
		\end{itemize}
	\end{block}
\end{frame}

\begin{frame}
	%	\frametitle{罗马数字}
	\begin{block}{右加左减}
		\begin{itemize}
			\item 在较大的罗马数字的右边记上较小的罗马数字,表示大数加小数
			\item 在较大的罗马数字的左边记上较小的罗马数字,表示大数减小数
			\item 左减的数字有限制,仅限于I、X、C
			\item 左减时不可跨越一个位值
		\end{itemize}
	\end{block}
	\begin{exampleblock}{加线乘千}
		\begin{itemize}
			\item 在罗马数字的上方加上一条横线,表示原数的1000倍
			\item 同理,如果上方有两条横线,即是原数的1000000倍
		\end{itemize}
	\end{exampleblock}
	\begin{alertblock}{数码限制}
		\begin{itemize}
			\item 同一数码最多只能连续出现三次
			\item 例外:有时可以用IIII代替IV
		\end{itemize}
	\end{alertblock}
\end{frame}

\begin{frame}\frametitle{罗马数字}
	\begin{columns}[c]
		\begin{column}{.45\textwidth}
			\begin{exampleblock}{罗马数字 -> 数值:}
				\begin{itemize}
					\item<2-> VIII
					\item<4-> XIV
					\item<6-> LX
					\item<8-> XCIX
					\item<10-> CXCIX
					\item<12-> M$\overline{\text{V}}$
				\end{itemize}
			\end{exampleblock}
		\end{column}
		\begin{column}{.45\textwidth}
			\begin{alertblock}{答案:}
				\begin{itemize}
					\item<3-> 8
					\item<5-> 14
					\item<7-> 60
					\item<9-> 99
					\item<11-> 199
					\item<13-> 4000
				\end{itemize}
			\end{alertblock}
		\end{column}
	\end{columns}
\end{frame}

\begin{frame}\frametitle{罗马数字}
\begin{columns}[c]
\begin{column}{.45\textwidth}
	\begin{exampleblock}{数值 -> 罗马数字:}
		\begin{itemize}
			\item<2-> 19
			\item<4-> 145
			\item<6-> 361
			\item<8-> 1437
			\item<10-> 8191
			\item<12-> 65537
		\end{itemize}
	\end{exampleblock}
\end{column}
\begin{column}{.45\textwidth}
	\begin{alertblock}{答案:}
		\begin{itemize}
			\item<3-> XIX
			\item<5-> CXLV
			\item<7-> CCCLXI
			\item<9-> MCDXXXVII
			\item<11-> $\overline{\text{V}}$MMMCXCI
			\item<13-> $\overline{\text{L}}\overline{\text{X}}\overline{\text{V}}$DXXXVII
		\end{itemize}
	\end{alertblock}
\end{column}
\end{columns}
\end{frame}