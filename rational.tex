\section{有理数}

\begin{frame}
	\begin{block}{}
		我们学过那些数?
	\end{block}\pause
	\begin{itemize}
		\item 自然数
		\item 整数
		\item 小数
		\item 分数
		\item $\pi$
	\end{itemize}
\end{frame}

\begin{frame}
	\begin{block}{}
		还记得小数的分类吗?
	\end{block}\pause
	\vspace{1ex}
	\(
	\text{小数}
	\begin{dcases}
	\text{有限小数} \hspace{9em} \tikzmark{Q top} \hspace{6em} \tikzmark{R top} \\
	 \text{无限小数} \pause
	\begin{dcases}
	\text{无限循环小数} \hspace{2em} \tikzmark{Q bottom} \\ 
	\text{无限不循环小数} \tikzmark{NQ} \hspace{7em} \tikzmark{R bottom}
	\end{dcases}
	\end{dcases}
	\)\pause
	\begin{tikzpicture}[overlay,remember picture]
		\VerticalBrace[thick, blue]{Q top}{Q bottom}{有理数}
	\end{tikzpicture} \pause
	\begin{tikzpicture}[overlay,remember picture]
		\RightArrow[thick,blue]{NQ}{无理数}
	\end{tikzpicture} \pause
	\begin{tikzpicture}[overlay,remember picture]
		\VerticalBrace[thick, blue]{R top}{R bottom}{实数}
	\end{tikzpicture}
\end{frame}

\begin{frame}
	\begin{block}{}
		能不能给``有理数''下个定义?
	\end{block}\pause
	注意:有限小数 $\to$ 整数或分数,无限循环小数 $\to$ 分数;\\ \pause
	\vspace{1ex}
	\qquad $\text{整数} = \dfrac{\text{整数}}{1}$,$\text{分数} = \dfrac{\text{整数}}{\text{整数}}$ \pause
	\begin{definition}{\alert{有理数}}
		能表示为整数之比的数。
	\end{definition} \pause
	\[\text{有理数}
		\begin{dcases}
			\text{整数} \uncover<6->{\quad 0, 1, 4, -3, \frac{6}{6}, \frac{40}{8}, 600\%, \cdots} \\
			\text{分数} \uncover<7->{\quad \frac{1}{2}, \frac{4}{6}, -\frac{7}{12}, 3.14, -0.05, \cdots}
		\end{dcases}
	\]
\end{frame}

\begin{frame}
	\begin{block}{}
		如何将有理数分类?
	\end{block} \pause
	\begin{columns}[c]
		\begin{column}{0.5\textwidth}
			\[\text{有理数}
				\begin{dcases}
					\text{整数} \uncover<3->{\begin{dcases} \text{正整数} \\ 0 \\ \text{负整数} \end{dcases}} \\
					\text{分数} \uncover<4->{\begin{dcases} \text{正分数} \\ \text{负分数} \end{dcases}}
				\end{dcases}
			\]
		\end{column} \pause
	\begin{column}{0.5\textwidth}
		\uncover<5->{\[\text{有理数}
			\begin{dcases}
				\text{正有理数} \uncover<6->{\begin{dcases} \text{正整数} \\ \text{正分数} \end{dcases}} \\
				0 \\
				\text{负有理数} \uncover<7->{\begin{dcases} \text{负整数} \\ \text{负分数} \end{dcases}}
			\end{dcases}
		\]}
	\end{column}
	\end{columns}
\end{frame}